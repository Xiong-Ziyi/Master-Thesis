\documentclass[a4paper]{article}
\usepackage[margin = 2.5cm]{geometry}
\usepackage[onehalfspacing]{setspace}
\usepackage{graphicx}
\usepackage{amsmath, amsthm, amssymb}
\usepackage[breaklinks=true,colorlinks=true,linkcolor=blue,urlcolor=blue,citecolor=blue]{hyperref}
\usepackage[T1]{fontenc}
\usepackage{placeins}
\usepackage[natbib,abbreviate=true,doi=false,style=numeric-comp,giveninits=true,sorting=none]{biblatex}
\usepackage[space-before-unit=true,per-mode = symbol]{siunitx}
\usepackage{float}

\setlength{\parindent}{0pt}
\setlength{\parskip}{0.5em} 

\title{Beam Shaper}
\author{Ziyi Xiong}
\date{\today}

\addbibresource{MyBibliography.bib}
\graphicspath{{../Figures/}}

\DeclareSIUnit{\dBm}{dBm}
\DeclareSIUnit[per-mode=reciprocal]\WN{\per\centi\meter}

\begin{document}

\maketitle
\tableofcontents

\newpage
\section{Keplerian Type Beam Shaper}

Simulation with parameters and lens data sheet given in \textit{Laser Beam Shaping Techniques}

\begin{figure}[H]
    \centering
    \includegraphics[width=0.75\textwidth]{121K_Parameters.png}
    \caption{Parameters of Keplerian type beam shaper system.}
\end{figure}

The parameters gives the apodization factor as: 

\begin{equation}
    G = \left(\frac{r_{max}}{\omega_0}\right)^2 \approx 2.93 
\end{equation}

\subsection{Initial system}

The lens data sheet given in \textit{Laser Beam Shaping Techniques}:

\begin{figure}[H]
    \centering
    \includegraphics[width=0.75\textwidth]{Keplerian_init.png}
    \caption{Initial Keplerian type beam shaper system.}
\end{figure}

This system gives a profile as below in Zemax OpticStudio:

\begin{figure}[H]
    \centering
    \includegraphics[width=0.75\textwidth]{Keplerian_init_profile.png}
    \caption{Initial beam profile at the output plane.}
\end{figure}

\subsection{Optimized with only conic constant}

The lens data sheet given in \textit{Laser Beam Shaping Techniques}:

\begin{figure}[H]
    \centering
    \includegraphics[width=0.75\textwidth]{Keplerian_opted_k.png}
    \caption{Optimized Keplerian type beam shaper system with only conic constant as variable.}
\end{figure}

In Zemax OpticStudio, the output profile given by this is:

\begin{figure}[H]
    \centering
    \includegraphics[width=0.75\textwidth]{Keplerian_opted_k_profile.png}
    \caption{Output beam profile at the output plane after optimization with only conic constant as variable.}
\end{figure}

\subsection{Optimized with up to 4th order}

The lens data sheet given in \textit{Laser Beam Shaping Techniques}:

\begin{figure}[H]
    \centering
    \includegraphics[width=0.75\textwidth]{Keplerian_opted_to_4th.png}
    \caption{Optimized Keplerian type beam shaper system with up to 4th order as variable.}
\end{figure}

In Zemax OpticStudio, the output profile given by this is:

\begin{figure}[H]
    \centering
    \includegraphics[width=0.75\textwidth]{Keplerian_opted_to_4th_profile.png}
    \caption{Output beam profile at the output plane after optimization with up to 4th order as variable.}
\end{figure}

\subsection{Optimized with up to 6th order}

The lens data sheet given in \textit{Laser Beam Shaping Techniques}:

\begin{figure}[H]
    \centering
    \includegraphics[width=0.75\textwidth]{Keplerian_opted_to_6th.png}
    \caption{Optimized Keplerian type beam shaper system with up to 6th order as variable.}
\end{figure}

In Zemax OpticStudio, the output profile given by this is:
\begin{figure}[H]
    \centering
    \includegraphics[width=0.75\textwidth]{Keplerian_opted_to_6th_profile.png}
    \caption{Output beam profile at the output plane after optimization with up to 6th order as variable.}      
\end{figure}

\newpage
\section{Galilean Type Beam Shaper}

Since the Galilean type beam shaper can not result in a 1:1 beam size conversion, for making a better comparison, the $R_max$ is timed by 3 to make the marginal ray get bent the same size of angle with respect to the optical axis but in an opposite direction.

So the parameters are modified as below:

\begin{equation}
    \omega_0 = 2.366 \quad R_{max} = 12.15 \quad r_{max} = 4.05 \quad n = 1.46071 \quad d = 150
\end{equation}

\subsection{Initial system}
Given the parameters above, the initial system's lens data can be derived using equations provided in \textit{Laser Beam Shaping Techniques}:


\end{document}