\documentclass[a4paper]{article}
\usepackage[margin = 2.5cm]{geometry}
\usepackage[onehalfspacing]{setspace}
\usepackage{graphicx}
\usepackage{amsmath, amsthm, amssymb}
\usepackage[breaklinks=true,colorlinks=true,linkcolor=blue,urlcolor=blue,citecolor=blue]{hyperref}
\usepackage[T1]{fontenc}
\usepackage{placeins}
\usepackage[natbib,abbreviate=true,doi=false,style=numeric-comp,giveninits=true,sorting=none]{biblatex}
\usepackage[space-before-unit=true,per-mode = symbol]{siunitx}
\usepackage{float}

\title{Beam Shaper}
\author{Ziyi Xiong}
\date{\today}

\addbibresource{MyBibliography.bib}
\graphicspath{{../Figures/}}

\DeclareSIUnit{\dBm}{dBm}
\DeclareSIUnit[per-mode=reciprocal]\WN{\per\centi\meter}

\begin{document}
\maketitle

\section{Keplerian Type Beam Shaper}
Simulation with parameters given in \textit{Laser Beam Shaping Techniques}

\subsection{Initial System}

\begin{figure}[H]
    \centering
    \includegraphics[width=0.8\textwidth]{Keplerian_init.png}
    \caption{Initial Keplerian type beam shaper system.}
\end{figure}

This system gives a profile as below:

\begin{figure}[H]
    \centering
    \includegraphics[width=0.6\textwidth]{Keplerian_init_profile.png}
    \caption{Initial beam profile at the output plane.}
\end{figure}

\section{Galilean Type Beam Shaper}

\end{document}